\documentclass[10pt,a4paper]{oblivoir}

\usepackage{fapapersize}
\usefapapersize{*,*,1in,*,1in,*}

\setmainfont{Noto Serif}
\setsansfont{Noto Sans}
\setmonofont{Noto Sans Mono}
\setkomainfont(Noto Serif CJK KR)(* Bold)(Noto Sans CJK KR)
\setkosansfont[Noto Sans CJK KR]()( Bold)( Medium)

\usepackage{tcolorbox}
\usepackage{lipsum}

\NewDocumentCommand \senv { m }
  {
    \cmd{\begin\{#1\}}
  }
\NewDocumentCommand \eenv { m }
  {
    \cmd{\end\{#1\}}
  }


\newcommand{\published}[1]{%
  \gdef\puB{#1}}
\renewcommand{\maketitlehooka}{%
  \par\begin{flushright}\puB\end{flushright}}

\begin{document}

\begin{titlingpage}
%%% TITLE
\pretitle{\begin{flushright}\HUGE\sffamily}
\title{The Memoir Class 번역}
\posttitle{\par\end{flushright}\vskip 1cm}

%%% AUTHOR
\preauthor{\begin{flushright}
           \large \lineskip 0.5em%
           \begin{tabular}[t]{c}}
\author{이재호 옮김}
\postauthor{\end{tabular}\par\end{flushright}\vskip 0.5cm}

%%% DATE
\predate{\begin{flushright}}
\date{마지막 수정일: \today}
\postdate{\par\end{flushright}}

% thanks에서 \\는 안되고 \par만 되는 이유?
\published{expl3 스터디 그룹%
           \thanks{\TeX을 배울 수 있는 자리 마련해주셔서 감사합니다.\par텍스트}\\
           2차 모임 (2019년 7월 13일)}

\thanksmarkseries{fnsymbol}
\thanksheadextra{(}{)}
% This is same as
% \thanksmarkstyle{(#1)}
\setlength{\thanksmarkwidth}{1em}
\setlength{\thanksmarksep}{-0.9em}

\setlength{\droptitle}{10.5cm}

\usethanksrule

\maketitle
\end{titlingpage}

\published{}
% \pretitle{\begin{center}\huge\sffamily}
% \posttitle{\par\end{center}\vskip 1cm}
\emptythanks

\maketitle

\renewcommand{\abstractname}{Abstract}
\renewcommand{\abstractnamefont}{\small\scshape}
\renewcommand{\abstracttextfont}{\small\sffamily}

% \abstractrunin
% \abslabeldelim{.}
% \setlength{\abstitleskip}{-4.5em}

% \renewcommand{\absnamepos}{flushleft}

\setlength{\absparindent}{0pt}
% \begin{onecolabstract}
\begin{abstract}
  \lipsum[1]
\end{abstract}
% \end{onecolabstract}

\chapter{타이틀}

표준 클래스들은 표지를 설정하는데 큰 도움이 되지 못하는데, \cmd{\maketitle}
명령은 테크니컬 논문집에서 논문의 제목을 생성하는 것이 주목적이기 때문이다.
이는 학위 논문, 보고서나 서적의 표지를 만들기에 불충분하다.
필자는 이를 위해 \cmd{\maketitle}을 무시하고, 직접 \LaTeX\ 표준 명령어를
사용해 표지 레이아웃을\footnote{만약 여러분이 학위 논문을 작성 중이라면, 그것이
어떻게 생겨야 하는지 아마도 정해져 있을 것이다.}
디자인할 것을 권장한다.

Ruari McLean을 인용하자면, 그는 표지에 대해서 다음과 같이 말한다:
\begin{quotation}
  표지는 책의 실제 제목과 (있을 경우 부제목을 포함해) 저자의 이름, 출판사,
  그리고 간혹 삽화의 개수를 포함하지만, 나아가 그것보다 많은 일을 해야 한다.
  디자이너의 관점에서 표지는 책에서 가장 중요한 부분으로, 책의 스타일을 정한다.
  표지는 독자와의 소통을 시작한다\ldots

  만약 책에서 삽화가 큰 비중을 차지한다면, 표지는 이를 시각적으로 보일 수
  있거나 그래야 한다.
  책에서, 예컨대 장의 시작에서라던지, 어떠한 형태로든 장식이 사용된다면 독자는
  이것이 표지에서도 반복될 것을 기대할 것이다.

  책의 형식이 어떻든지간에, 표지는 그것의 맛보기를 보여줘야 한다.
  만약 책이 줄글로 구성되어 있다면, 표지는 적어도 그것과 조화를 이룰 수 있어야
  한다.
  표지 자체는 글 영역의 너비보다 넓어서는 안되며, 일반적으로는 더 좁을
  것이다\ldots
\end{quotation}

McLean의 표지를 모방한 것이 \fref{figure:titleTH}에 나와 있다.

\begin{figure}
\centering
% \begin{showtitle}
% \titleTH
% \end{showtitle}
\caption{Layout of a title page for a book on typography}\label{figure:titleTH}
\end{figure}

\cmd{\maketitle} 명령의 조판 형식은 \LaTeX 표준 클래스에서 사실상 고정된
것으로 보아야 한다.
이 클래스는 제목 정보, 즉 \cmd{\title}, \cmd{\author}, \cmd{\date}의 내용이
표시되는 형태를 수정할 수 있는 일련의 형식화 명령들을 제공한다.
또, 이는 위 명령들의 값을 나중에 문서에서 다시 사용할 수 있도록 유지한다.
나아가 이 클래스는 \cmd{\maketitle} 명령이 사용된 후에 사용된 명령 값이
자동으로 지워지는 것을 방지한다.
그러므로 하나의 문서 안에서 같거나 다른 표지가, 예를 들면 반표지와 전표지에,
여러 번 나오게 할 수도 있다.
\cmd{\thanks} 명령은 기능이 확장되어 감사의 말 주석의 표지 부호와 레이아웃을
다양하게 설정할 수 있게 되었다.

일반적으로, \cmd{\maketitle} 레이아웃에서 약간의 변화를 넘어선다면
\cmd{\maketitle}을 무시하고 레이아웃 전체를 여러분이 손수 만들어 여러분이
종의의 원하는 곳에 모든 것을 위치시키는 편이 낫다.

\begin{figure}
\centering
% \begin{showtitle}
% \titleDS
% \end{showtitle}
\caption{Example of a mandated title page style for a doctoral thesis}\label{figure:titleDS}
\end{figure}

\begin{figure}%\setlength{\unitlength}{1pt}
\centering
%\begin{showtitle}
% {\titleRB}
%\end{showtitle}
\caption{Example of a Victorian title page}\label{figure:titleRB}
\end{figure}

\begin{figure}
\centering
% \begin{showtitle}
% \titleDB
% \end{showtitle}
\caption{Layout of a title page for a book on book design}\label{figure:titleDB}
\end{figure}

\begin{figure}
\centering
% \begin{showtitle}
% \titleGM
% \end{showtitle}
\caption{Layout of a title page for a book about books}\label{figure:titleGM}
\end{figure}

\section{표지 양식화하기}

표지를 조판하기 위해 제공되는 도구들은 국한되어 있는데, 이들은 본질적으로
테크니컬 논문집에 출판되는 논문의 제목 형식을 제공해주는 것이 전부이다.
이들은 보고서의 제목을 조판할 때에도 빠르고 지저분한 방법을 제공해줄 수 있지만,
책이나 학위 논문과 같은 중요한 작업의 표지에 대해서는 수작업이 필요하다.
예컨대 나의 학생인 Donal Sanderson은 그의 박사 학위 논문을 조판하기 위해
\LaTeX\ 을 사용했고, \fref{figure:titleDS}에 Rensselear Polytechnic
Institute에서 1994년에 지정된 표지 양식이 나와 있다.
이외의 다양한 표지의 예시와, 이들을 만드는데 사용된 코드가 \cite{TITLEPAGES}에
있다.

\cite{TITLEPAGES}에 나오는 또 다른 수작업된 표지가 \fref{figure:titleRB}에 나와
있다.
이것은 필자가 19세기 말 즈음에 출판된 옛 소책자를 기반으로 한 것이며, 다양한
활자를 표시할 수 있는 빅토리아 시대의 프린터에 대한 애정을 확인할 수 있다.
여기서 괘선이 표지의 핵심적인 부분이라고 할 수 있다.
본 매뉴얼을 위해 필자는 \LaTeX\ 표준 배포판에 포함된 New Century Schoolbook
글꼴을 사용했는데, 본래는 SoftMaker/ATF 라이브러리에서 라이선스를 받은, 그리고 
Christoper League의 공로로 \LaTeX\ 을 지원하는 FontSite 글꼴 중 하나인 Century
Old Style을 선택했었다.

\fref{figure:titleDB}의 표지는 \textit{The Design of
Books}~\cite{ADRIANWILSON93}의 양식을 따르며, 페이지는 Nicholas Basbanes의
\textit{A Gentle Madness: Bibliophiles, Bibliomanes, and the Eternal Passion
for Books}와 유사한 것이 \fref{figure:titleGM}에 나와 있다.
이들 모두는 \cite{TITLEPAGES}에서 가져온 것이며 수작업된 것들이다.

반면에 다음 코드는 \cmd{\maketitle}의 표준 레이아웃을 생성한다.
% \begin{egsource}{eg:maketitle}
% \title{MEANDERINGS}
% \author{T. H. E. River \and
%         A. Wanderer\thanks{Supported by a grant from the
%         R. Ambler's Fund}\\
%         Dun Roamin Institute, NY}
% \date{1 April 1993\thanks{First drafted on 29 February 1992}}
% ...
% \maketitle
% \end{egsource}

본 클래스의 이 부분은 \textsf{titling} 패키지를 재구현한 것이다.

이 클래스는 설정 가능한 \cmd{\maketitle} 명령을 제공한다.
\cmd{\maketitle} 명령은 이 클래스에서 본질적으로
\begin{verbatim}
\newcommand{\maketitle}{%
   \vspace*{\droptitle}
   \maketitlehooka
   {\pretitle \title \posttitle}
   \maketitlehookb
   {\preauthor \author \postauthor}
   \maketitlehookc
   {\predate \date \postdate}
   \maketitlehookd
   \thispagestyle{title}
}
\end{verbatim}
와 같이 정의되는데, 이때 \textsl{title} 페이지 스타일은 처음에는 \textsf{plain}
페이지 스타일과 동일하다.
\cmd{\maketitle} 내에서 사용되는 각종 매크로는 아래에 설명되어 있다.

\begin{tcolorbox}
\cmd{\pretitle}\marg{text} \cmd{\posttitle}\marg{text}\\
\cmd{\preauthor}\marg{text} \cmd{\postauthor}\marg{text}\\
\cmd{\predate}\marg{text} \cmd{\postdate}\marg{text}
\end{tcolorbox}
위 여섯 개의 명령들은 각자 하나의 인자 \marg{text}를 가지며, 이는 문서에서
\cmd{\maketitle} 명령의 표준 요소의 조판을 통제한다.
\cmd{\title}은 본질적으로 \cmd{\pretitle}과 \cmd{\posttitle} 사이에서
처리되는데, 다음과 같이
\begin{verbatim}
{\pretitle \title \posttitle}
\end{verbatim}
되며 \cmd{\author}과 \cmd{\date} 명령에 대해서도 유사하다.
이 명령들은 \textsf{report} 클래스의 일반적인 \cmd{\maketitle}의 조판 결과를
모방하도록 초기화되어 있다.
즉, 명령들의 기본 정의는 다음과 같다.
\begin{verbatim}
\pretitle{\begin{center}\LARGE}
\posttitle{\par\end{center}\vskip 0.5em}
\preauthor{\begin{center}
           \large \lineskip 0.5em%
           \begin{tabular}[t]{c}}
\postauthor{\end{tabular}\par\end{center}}
\predate{\begin{center}\large}
\postdate{\par\end{center}}
\end{verbatim}

이들은 다른 효과를 내기 위해 바꿀 수 있다.
예를 들어 우측 정렬된 sans-serif 제목과 좌측 정렬된 small caps 날짜를 얻으려면
다음과 같이 하라.
\begin{verbatim}
\pretitle{\begin{flushright}\LARGE\sffamily}
\posttitle{\par\end{flushright}\vskip 0.5em}
\predate{\begin{flushleft}\large\scshape}
\postdate{\par\end{flushleft}}
\end{verbatim}

\begin{tcolorbox}
\cmd{\droptitle}
\end{tcolorbox}
\cmd{\maketitle} 명령은 제목을 페이지의 특정 높이에 놓는다.
여러분은 제목의 세로 위치를 \cmd{\droptitle} 길이를 통해 변경할 수 있다.
여기에 양수를 대입하면 제목을 낮출 것이며, 음수를 대입하면 높일 것이다.
기본 정의는 다음과 같다.
\begin{verbatim}
\setlength{\droptitle}{0pt}
\end{verbatim}

\begin{tcolorbox}
\cmd{\maketitlehooka} \cmd{\maketitlehookb}\\
\cmd{\maketitlehookc} \cmd{\maketitlehookd}
\end{tcolorbox}
이 네 후크 명령은 \cmd{\maketitle}에 추가적인 요소를 넣을 수 있도록 제공된다.
이들은 기본적으로 아무것도 하지 않도록 정의되어 있지만, 재정의할 수 있다.
예를 들어, 일부 출판물에서는 논문이 출판된 곳에 대한 문장을 실제 제목 문구
직전에 삽입하는 것을 요구한다.
다음은 \cmd{\published} 명령을 정의하여 출판 정보를 담을 수 있도록 하며,
\cmd{\maketitle}에 의해 자동적으로 출력된다.
\begin{verbatim}
\newcommand{\published}[1]{%
   \gdef\puB{#1}}
\newcommand{\puB}{}
\renewcommand{\maketitlehooka}{%
   \par\noindent \puB}
\end{verbatim}
이후
\begin{verbatim}
\published{Originally published in 
          \textit{The Journal of ...}\thanks{Reprinted with permission}}
...
\maketitle
\end{verbatim}
와 같이 하여 출판 정보와 일반적인 제목 정보를 모두 출력할 수 있다.
새로운 \cmd{\published} 명령과 함께 \cmd{\thanks} 명령어를 사용하기 위해서
추가적인 조치가 필요 없다는 것에 주의하라.

\begin{tcolorbox}
\senv{titlingpage} text \eenv{titlingpage} \\
\senv{titlingpage*} text \eenv{titlingpage*}\\
\cmd{\titlingpageend}\marg{twoside code}\marg{oneside code} 
\end{tcolorbox}
표준 클래스들과 \textsf{titlepage} 옵션이 함께 사용되면, \cmd{\maketitle}은
제목 요소를 숫자가 붙지 않은 페이지에 넣고 새로운 페이지를 페이지 번호 1로
시작한다.
표준 클래스는 \cmd{titlepage} 환경도 제공하여, 번호가 붙지 않은 새로운 장을
시작하고 이후 다시 페이지 번호 1부터 새로운 페이지를 시작한다.
이 표지에 어떤 내용을 넣고 어디에 놓을지는 전적으로 여러분의 책임에 달려 있다.
만약 \cmd{\maketitle}이 \cmd{titlepage} 환경 안에서 사용된다면 이는 또 다른
페이지를 시작할 것이다.

이 클래스는 \textsf{titlingpage} 옵션이나 \cmd{titlepage} 환경 둘 중 어느 것도
제공하지 않는다.
대신 이는 \textsf{titlepage} 옵션과 \cmd{titlepage} 명령 중간쯤 되는
\cmd{titlingpage} 환경을 제공한다.
여러분은 \cmd{titlingpage} 환경에서 \cmd{\maketitle}을 포함한 명령들을 사용할
수 있다.
\textsf{titlingpage} 페이지 스타일이 사용되며, 끝에는 번호 1이 붙은 일반적인
페이지를 시작한다 (\senv{titlingpage*}는 페이지 번호를 재설정하지 않는다).
\textsl{titlingpage} 페이지 스타일은 \textsl{empty} 페이지 스타일과 같게 초기
설정되어 있다.

\cmd{titlingpage}의 끝에는 지우기 코드가 실행되는데, 이를 통해 다음 페이지 혹은
다음 우츨 페이지로 이동할 수 있다.
\cmd{\titlingpageend{\marg{twoside code}}{\marg{oneside code}}}를 사용해서
지우기 코드를 지정할 수 있다.
기본값은 각각 \cmd{\cleardoublepage}와 \cmd{\clearpage}이다.\footnote{따라서 이
수정은 기존의 문서를 바꾸지 않을 것이다, LM, 2018/03/06.}
그러나 그냥 \cmd{\clearforchapter}을 따르도록 하는 것이 더 나은 선택이 될 수
있다.
\begin{verbatim}
\titlingpageend{\clearforchapter}{\clearforchapter}
\end{verbatim}
이 값을 사용하면, \cmd{titlingpage}는 \textsf{openany}와 함께 예상대로 작동할
것이다.

예를 들어, \textsl{plain} 페이지 스타일로 제목과 요약을 같은 표지에 두고 싶다면
다음과 같이 한다.
\begin{verbatim}
\begin{document}
\begin{titlingpage}
\aliaspagestyle{titlingpage}{plain}
\setlength{\droptitle}{30pt} lower the title
\maketitle
\begin{abstract}...\end{abstract}
\end{titlingpage}
\end{verbatim}

그러나 \senv{titlingpage}를 사용해서 표지를 만드는 것이 필수는 아니므로,
여러분은 특수한 환경 없이 일반적인 \LaTeX\ 조판을 사용할 수 있다.
그 방법은 다음과 같을 것이다.
\begin{verbatim}
\pagestyle{empty}
%%% Title, author, publisher, etc.,  here
\cleardoublepage
...
\end{verbatim}

기본적으로, 제목 정보는 편집 영역 너비를 기준으로 가운데 정렬된다.
간혹 누군가 \texttt{comp.text.tex} 뉴스 그룹에 표지의 제목 정보를 실물 페이지
기준으로 가운데 정렬하는 방법을 묻고는 한다.
만약 편집 영역이 실제 페이지 기준으로 가운데에 있다면 기본 가운데 정렬로 충분할
것이다.
만약 편집 영역이 실제로 가운데가 아니라면, 제목 정보를 가로 방향으로
이동시키던가, \cmd{\maketitle}에게 편집 영역이 가로 방향으로 이동되었다고
믿게할 수 있다.
가장 간단한 해결책은 \cmd{\calccentering} 명령과 \cmd{adjustwidth*} 환경을
사용하는 것이다.
예를 들면 다음과 같다.
\begin{verbatim}
\begin{titlingpage}
  \calccentering{\unitlength}
  \begin{adjustwidth*}{\unitlength}{-\unitlength}
    \maketitle
  \end{adjustwidth*}
\end{titlingpage}
\end{verbatim}

\begin{tcolorbox}
\cmd{\title}\marg{text} \cmd{\thetitle}\\
\cmd{\author}\marg{text} \cmd{\theauthor}\\
\cmd{\date}\marg{text} \cmd{\thedate}
\end{tcolorbox}

일반적인 문서 클래스에서 \cmd{\maketitle}을 위해 사용되는 \cmd{\title},
\cmd{\author}과 \cmd{date} 매크로의 내용들(\meta{text})은 \cmd{\maketitle} 사용
이후 접근할 수 없다.
본 클래스는 \cmd{\thetitle}, \cmd{\theauthor}과 \cmd{\thedate} 명령을 제공하여
원한다면 이후 문서에서 제목 요소를 출력하는데 사용할 수 있도록 한다.

\begin{tcolorbox}
\cmd{\and} \cmd{\andnext}
\end{tcolorbox}
매크로 \cmd{\and}는 \cmd{\author} 명령어의 인자 안에 쓰이며, 저자들의 이름
사이에 추가 공백을 넣어준다.
이 클래스의 \cmd{\andnext} 매크로는 공백 대신에 새로운 줄을 넣어준다.
\cmd{\theauthor} 매크로 안에서 \cmd{\and}와 \cmd{\andnext}는 둘 다 쉼표로
치환된다.

본 클래스는 표준 클래스처럼 \cmd{\maketitle} 사용 이후에 표지 명령을 자동으로
끄는 관행을 따르지 않는다.
여러분은 원한다면 문서에 여러 개의 \cmd{\title}, \cmd{\author}, \cmd{date},
그리고 \cmd{\maketitle} 명령들을 넣을 수 있다.
예를 들어, 일부 보고서는 표지로 시작해서 뒤이어 요약문이 따르고, 다르게 표현될
수도 있는 또다른 제목을 본문 앞에 가진다.
이는 다음과 같이 구현할 수 있다.
\begin{verbatim}
\title{Cover title}
...
\begin{titlingpage}
\maketitle
\end{titlingpage}
...
\title{Body title}
\maketitle
...
\end{verbatim}

\begin{tcolorbox}
\cmd{\killtitle} \cmd{\keepthetitle}\\
\cmd{\emptythanks}
\end{tcolorbox}
\cmd{\killtitle} 매크로 사용 후에 \cmd{\thetitle} 등을 포함한 모든 표지 관련
명령을 사용하지 못하도록  한다 (이 명령을 사용해 \cmd{\thetitle} 등과 같은
명령이 필요하지 않을 경우 매크로 공간을 확보할 수 있다).
이 명령은 본 클래스에서 표준 클래스의 자동 종료 수행의 수동 버전이다.
\cmd{\keepthetitle} 명령은 비슷한 기능을 하지만, 다른 모든 기능은 끄면서
\cmd{\thetitle}, \cmd{\theauthor}과 \cmd{\thedate} 명령을 유지한다는 차이가
있다.

\cmd{\emptythanks} 명령은 기존 \cmd{\thanks}의 모든 문구를 지운다.
이 명령은 \cmd{\maketitle}이 여러번 사용될 경우 유용하다.
\cmd{\thanks} 명령은 매 사용마다 문구를 쌓아올리므로, \cmd{\maketitle}을 매번
사용할 때 기존의 모든 \cmd{\thanks} 문구가 새 문구와 함께 출력될 것이다.
이를 방지하기 위해서는 \cmd{\maketitle}을 사용하기 전에 매번
\cmd{\emptythanks}를 넣으면 된다.


\section{감사의 말 양식화하기}

본 클래스는 설정 가능한 \cmd{\thanks} 명령을 제공한다.
\begin{tcolorbox}
\cmd{\thanksmarkseries}\marg{format}\\
\cmd{\symbolthanksmark}
\end{tcolorbox}
모든 \cmd{\thanks}는 표지나 각주에 기호로 표시된다.
명령 \cmd{\thanksmarkseries}는 이러한 표지 양식을 바꾸는데 사용될 수 있다.
\meta{format} 인자는 카운터를 출력하는 형식들 중 하나의 이름이다.
이름은 카운터 형식과 같지만 백슬래시를 포함하지 않는다.
\cmd{\thanks}가 기호 대신에 소문자로 표시되기 위해서는 다음과 같이 하라.
\begin{verbatim}
\thanksmarkseries{alph}
\end{verbatim}
편의를 위해서 \cmd{\symbolthanksmark}는 목록을 각주 기호로 지정한다.
이 클래스를 사용하면 \meta{format}에는 \texttt{arabic}, \texttt{roman},
\texttt{Roman}, \texttt{alph}, \texttt{Alph} 그리고 \texttt{fnsymbol}이
이름으로 올 수 있다.

\begin{tcolorbox}
\cmd{\continuousmarks}
\end{tcolorbox}
\cmd{\thanks} 명령은 \cmd{footnote} 카운터를 사용하며, 보통 카운터는 표제 이후
영으로 설정되어 각자 마크는 다시 1부터 시작하게 된다.
만약 카운터가 영으로 초기화되면 안 된다면, \cmd{\continuousmarks}를 지시하라.
이는 감사의 말 표지로 숫자를 사용한다면 필요할 수 있다.

\begin{tcolorbox}
\cmd{\thanksheadextra}\marg{pre}\marg{post}
\end{tcolorbox}
\cmd{\thanksheadextra} 명령은 \meta{pre}와 \meta{post}를 각각 감사의 말 표제
이전과 이후에 표제 영역에 넣을 것이다.
기본적으로 \meta{pre}와 \meta{post}는 비어 있다.
예를 들어, 표제 표지를 괄호로 감싸라면 다음과 같이 하라.
\begin{verbatim}
\thanksheadextra{(}{)}
\end{verbatim}

\begin{tcolorbox}
\cmd{\thanksmark}\marg{n}
\end{tcolorbox}
간혹 동일한 감사의 말을, 예컨대 여섯 저자 중 네 명에게, 적용시켜야 할 경우, 

\end{document}
